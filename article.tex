\documentclass[a4paper,10pt]{article}
\usepackage[utf8]{inputenc}
\usepackage[english]{babel}
\usepackage{amsmath}
\usepackage{hyperref}
\begin{document}
\textbf{Scientifical ideas based on $E=mc^{2}$, Albert Einstein, relativity:}

\thispagestyle{empty}

The time $t$ is relative to the space and it is not absolute, so it is named 'ordinary time' and takes also the role of one actually 'relative' time variable into
the 4D 4-vector frame of reference $(x,y,z,t)$, in the fourth position.

The time $t$ is considered at the macroscopic level the independent temporal variable.

Under some relativistic considerations, here we consider the energy as a function of the 4D space and the mass, macroscopic, function of the ordinary 3D space; moreover, since $c$ is positive, we can write that:
$E=mc^{2}$ $\Rightarrow$ $c= \sqrt{\frac{E(x,y,z,t)}{{m}(x,y,z)}}$ and using the partial derivative with respect to $t$ we can then obtain that, for the mathematical rule of the derivative of a composite function: 
$0 = {\frac{1}{{2}\sqrt{{\frac{{E}(x,y,z,t)}{m(x,y,z)}}}}}
{\frac{\partial}{\partial t} }
{ \Big( {\frac{{E}(x,y,z,t)}{m(x,y,z)}} \Big) }$
and bringing out the mass, that is considered here function of the three ordinary space variable $x, y, z$ (in the form: ${m}(x,y,z)$), from the square root and out of the partial derivative (since the mass is constant, for hypothesis, with respect to the ordinary time), we have that:
$\Rightarrow 0=\frac{\frac{\partial {E}(x,y,z,t)}{\partial t}}{2m(x,y,z) \sqrt{ \frac{{E}(x,y,z,t)}{m(x,y,z)} }}$ .

Finally integrating with respect to the $t$ variable:

$0=\int^{t=t_2}_{t=t_1}{\frac{\frac{\partial {E}(x,y,z,t)}{\partial t}}{2{m}(x,y,z) \sqrt{ \frac{{E}(x,y,z,t)}{m(x,y,z)} }}}dt$
$\Rightarrow$
$0=\frac{1}{2\sqrt{{m}(x,y,z)}}$
$\int^{t=t_2}_{t=t_1}{\frac{\frac{\partial {E}(x,y,z,t)}{\partial t}}{\sqrt{ {E}(x,y,z,t) }}} dt$ .

Since the mass is always positive and, when it is present, it has a value that is never equal to zero, we have that:
$\Rightarrow$
$0=\int^{t_2}_{t_1}{\frac{\frac{\partial {E}(x,y,z,t)}{\partial t}}{\sqrt{ {E}(x,y,z,t) }}}dt$ ; since the partial derivative
is, generally speaking, a function of the time and so it is necessarily integrable in the independent (independent from the 3D space coordinates $x, y, z$) ordinary differential of the time $dt$,
this remarkable relativistic integral of the energy result is valid.

Since we can calculate with precision the time, if it could be possible to calculate with precision the energy contained for simplicity into one infinitesimal cube 
of side $l$ and volume $dx \cdot dy \cdot dz$, with $dx=l, dy=l, dz=l$ present physically in the ordinary 3D space for a particular time instant,
the variables and the functions that will appear in the solution of the indefinite integral for a given energy function ${E}(x,y,z,t)$,
that has been opportunely set up for the experiment, or that for some reason is known,
having solved the integral in the infinitesimal $dt$ (the differential of the time), could be calculated.
For a given known time instant $t=t_1$, that could be in example set to the beginning of the experiment (so, i.e. $t_1=0$ for simplicity), 
and for the time instant $t=t_2$, that denotes the end of the experiment, we could obtain more observations of the 4D energy values and 
in this way we could obtain a system of equations that has for result the values for the variables that still remain.

\textbf{Author: Amos Tibaldi, town of Reggiolo (R.E.), Italy. 13 march 2021, Saturday.}

\href{www.amos-tibaldi.it}{www.amos-tibaldi.it}

\end{document}
